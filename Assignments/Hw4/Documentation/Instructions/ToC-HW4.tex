\documentclass[letterpaper,11pt,twoside]{article}
\usepackage[margin=0.88in]{geometry}

%%%%%%%%%%%%%%%%%%%%%%%%%%%%%%%%%%%%%%%%%%%%%%%%%%%%%%%%%%%%%%%%%%%%%%%%%%%%%%%

\def\mypdfauthor{Dimitris Diochnos}
\def\mypdfsubject{Undergraduate Course}
\def\mypdftitle{CS 3823 - Theory of Computation, 2025F: HW4}
\def\myheadertitle{CS 3823 - Theory of Computation: Homework Assignment 4}
\def\mytitle{CS 3823 - Theory of Computation: Homework Assignment 4}
\def\thecurrentsemester{Fall 2025}
\def\myduedate{\textbf{Due:} Friday, November 14, 2025}

%%%%%%%%%%%%%%%%%%%%%%%%%%%%%%%%%%%%%%%%%%%%%%%%%%%%%%%%%%%%%%%%%%%%%%%%%%%%%%%

\usepackage{lmodern}      % I need this in order to present the tilde
\usepackage[T1]{fontenc}  % I need this in order to present the tilde in the middle of the line
\usepackage{xspace}
\usepackage{enumitem}
\usepackage[numbers,square,sort&compress]{natbib}
\usepackage[euler-digits, T1]{eulervm}
\usepackage{upgreek}
\usepackage[dvipsnames]{xcolor}
\usepackage[
   colorlinks%
   ,plainpages=false%This forces a unique identification of pages.
   ,hypertexnames=true%This is necessary to have exact link on Index page.
   ,naturalnames
   ,hyperindex
   ,citecolor=OliveGreen
   ,urlcolor=RoyalBlue
   ,pdfauthor={\mypdfauthor}
   ,pdftitle={\mypdftitle}
   ,pdfsubject={\mypdfsubject}
   %,pdfkeywords={...}
]{hyperref}
\usepackage{psfrag}
\usepackage{graphicx}
\usepackage{multirow}
\usepackage{multicol}
\usepackage{enumitem}
\usepackage{fancyhdr}
\usepackage{lastpage}
\usepackage{subcaption}
%%%%%%%%%%%%%%%%%%%%%%%%%%%%%%%%%%%%%%%%%%%%%%%%%%%%%%%%%%%%%%%%%%%
\usepackage{pstricks}
\usepackage{pst-func}
\psset{unit=0.1cm}%
\psset{linewidth=0.5}%
\psset{fillstyle=solid}%
%%%%%%%%%%%%%%%%%%%%%%%%%%%%%%%%%%%%%%%%%%%%%%%%%%%%%%%%%%%%%%%%%%%

\pagestyle{fancy}
\setlength{\headheight}{14pt}
\fancypagestyle{firststyle}
{
   \fancyhf{}
   \fancyfoot[L]{\today}
   \fancyfoot[R]{\thepage/\pageref*{LastPage}}
}
\newcommand{\stylishPagesColor}{Gray}
\newcommand{\stylishHref}[2]%
{\hypersetup{urlcolor=\stylishPagesColor}%
\href{#1}{#2}%
\hypersetup{urlcolor=RoyalBlue}}
\fancyhead{} % clear all header fields
\fancyhead[CO,CE]{\textsc{\myheadertitle}}
\fancyfoot{} % clear all footer fields
%\fancyfoot[LO,RE]{\today}
\fancyfoot[LO,RE]{November 2, 2025}
%\fancyfoot[CO,CE]{\thepage/\pageref*{LastPage}}
\fancyfoot[RO,LE]{\thepage/\pageref*{LastPage}}
\renewcommand{\headrulewidth}{0.0pt}
\renewcommand{\footrulewidth}{0.0pt}

%%%%%%%%%%%%%%%%%%%%%%%%%%%%%%%%%%%%%%%%%%%%%%%%%%%%%%%%%%%%%%%%%%%

% So that absolute values and norms are neat.
\usepackage{amsmath}
\providecommand{\abs}[1]{\lvert#1\rvert}
\providecommand{\norm}[1]{\lVert#1\rVert}
% Sets
\providecommand{\set}[1]{\ensuremath{\left\{#1\right\}}\xspace}
\providecommand{\powerset}[1]{\ensuremath{\mathcal{P}\left( #1\right)}\xspace}

%%%%%%%%%%%%%%%%%%%%%%%%%%%%%%%%%%%%%%%%%%%%%%%%%%%%%%%%%%%%%%%%%%%

\newcommand{\emptysymbol}{\ensuremath{\varepsilon}}
\newcommand{\emptystring}{\ensuremath{\varepsilon}}
\newcommand{\regexpunion}{\ensuremath{\cup}\xspace}

%%%%%%%%%%%%%%%%%%%%%%%%%%%%%%%%%%%%%%%%%%%%%%%%%%%%%%%%%%%%%%%%%%%

\def\jflapurl{http://www.jflap.org}

%%%%%%%%%%%%%%%%%%%%%%%%%%%%%%%%%%%%%%%%%%%%%%%%%%%%%%%%%%%%%%%%%%%

\usepackage{color}
\definecolor{RoyalBlue}{cmyk}{1, 0.50, 0, 0}
\definecolor{ForestGreen}{cmyk}{0.864, 0.0, 0.429, 0.396}
\definecolor{Brown}{cmyk}{0.0,0.692,0.925,0.529}

\newcommand{\WriteRed}[1]{{\color{red} #1 }\xspace}
\newcommand{\WriteRoyalBlue}[1]{{\color{RoyalBlue} #1 }\xspace}
\newcommand{\WriteForestGreen}[1]{{\color{ForestGreen} #1 }\xspace}
\newcommand{\WriteBrown}[1]{{\color{Brown} #1 }\xspace}
\newcommand{\WriteCustomColor}[1]{{\color{blue} #1 }\xspace}
\newcommand{\WriteSolutions}[1]{\WriteCustomColor{#1}}

%%%%%%%%%%%%%%%%%%%%%%%%%%%%%%%%%%%%%%%%%%%%%%%%%%%%%%%%%%%%%%%%%%%%%%%%%%%%%%%%%%%%%%%%%%%%%%%%%%%
%%%%%%%%%%%%%%%%%%%%%%%%%%%%%%%%%%%%%%%%%%%%%%%%%%%%%%%%%%%%%%%%%%%%%%%%%%%%%%%%%%%%%%%%%%%%%%%%%%%
%%%%%%%%%%%%%%%%%%%%%%%%%%%%%%%%%%%%%%%%%%%%%%%%%%%%%%%%%%%%%%%%%%%%%%%%%%%%%%%%%%%%%%%%%%%%%%%%%%%
%%%%%%%%%%%%%%%%%%%%%%%%%%%%%%%%%%%%%%%%%%%%%%%%%%%%%%%%%%%%%%%%%%%%%%%%%%%%%%%%%%%%%%%%%%%%%%%%%%%


\begin{document}

%%%%%%%%%%%%%%%%%%%%%%%%%%%%%%%%%%%%%%%%%%%%%%%%%%

\author{
\textsc{\thecurrentsemester} \hspace{3cm}\myduedate
}
\title{\mytitle}
\date{}
\maketitle

%%%%%%%%%%%%%%%%%%%%%%%%%%%%%%%%%%%%%%%%%%%%%%%%%%

\thispagestyle{firststyle}

%%%%%%%%%%%%%%%%%%%%%%%%%%%%%%%%%%%%%%%%%%%%%%%%%%

\vspace{-0.5cm}
\noindent\makebox[\linewidth]{\rule{\columnwidth}{2pt}}

%\section*{General Information}
\noindent\textbf{Related Reading.} 
Sections 2.2, 2.3. Chapter 3.\\
%Sections 8.1, 9.1, 9.2, 9.3\\
\noindent\textbf{Instructions.} Near the top of the first page of your solutions please list clearly \textbf{all} the members of the group (\underline{please see the syllabus for the collaboration policy}) who have created the solutions that you are submitting. Listing the names of the people in the group implies their full name and their 4x4 IDs.
Alternatively, you can use the space below and provide the relevant information 
in case you submit the solutions using this document.\\ 
\noindent\makebox[\linewidth]{\rule{\columnwidth}{2pt}}




\begin{center}
\textbf{Student Information for the Solutions Submitted}
\end{center}

\begin{center}
\begin{tabular}{c|c|c|}\cline{2-3}
 & Lastname, Firstname
 & 4x4 ID (e.g., dioc0000)
 \\\hline
%\multicolumn{1}{|c|}{\multirow{2}{*}{\textbf{1}}}
\multicolumn{1}{|c|}{\multirow{2}{*}{1}}
& \phantom{ABCDEFGHIJKLMNOPQRSTUVWXYZ0123456789} & \phantom{dioc0000dioc0000} \\
\multicolumn{1}{|c|}{}
           & \phantom{ABCDEFGHIJKLMNOPQRSTUVWXYZ0123456789} & \phantom{dioc0000dioc0000} \\\hline
%\multicolumn{1}{|c|}{\multirow{2}{*}{\textbf{2}}}
\multicolumn{1}{|c|}{\multirow{2}{*}{2}}
& \phantom{ABCDEFGHIJKLMNOPQRSTUVWXYZ0123456789} & \phantom{dioc0000dioc0000} \\
\multicolumn{1}{|c|}{}
           & \phantom{ABCDEFGHIJKLMNOPQRSTUVWXYZ0123456789} & \phantom{dioc0000dioc0000} \\\hline
%\multicolumn{1}{|c|}{\multirow{2}{*}{\textbf{3}}}
\multicolumn{1}{|c|}{\multirow{2}{*}{3}}
& \phantom{ABCDEFGHIJKLMNOPQRSTUVWXYZ0123456789} & \phantom{dioc0000dioc0000} \\
\multicolumn{1}{|c|}{}
           & \phantom{ABCDEFGHIJKLMNOPQRSTUVWXYZ0123456789} & \phantom{dioc0000dioc0000} \\\hline
%\multicolumn{1}{|c|}{\multirow{2}{*}{\textbf{4}}}
\multicolumn{1}{|c|}{\multirow{2}{*}{4}}
& \phantom{ABCDEFGHIJKLMNOPQRSTUVWXYZ0123456789} & \phantom{dioc0000dioc0000} \\
\multicolumn{1}{|c|}{}
           & \phantom{ABCDEFGHIJKLMNOPQRSTUVWXYZ0123456789} & \phantom{dioc0000dioc0000} \\\hline
%\multicolumn{1}{|c|}{\multirow{2}{*}{\textbf{5}}}
\multicolumn{1}{|c|}{\multirow{2}{*}{5}}
& \phantom{ABCDEFGHIJKLMNOPQRSTUVWXYZ0123456789} & \phantom{dioc0000dioc0000} \\
\multicolumn{1}{|c|}{}
           & \phantom{ABCDEFGHIJKLMNOPQRSTUVWXYZ0123456789} & \phantom{dioc0000dioc0000} \\\hline
\multicolumn{1}{|c|}{\multirow{2}{*}{6}}
& \phantom{ABCDEFGHIJKLMNOPQRSTUVWXYZ0123456789} & \phantom{dioc0000dioc0000} \\
\multicolumn{1}{|c|}{}
           & \phantom{ABCDEFGHIJKLMNOPQRSTUVWXYZ0123456789} & \phantom{dioc0000dioc0000} \\\hline
\end{tabular}
\end{center}

\thispagestyle{empty}

%\medskip

%\setlength{\columnseprule}{2pt}
%\def\columnseprulecolor{\color{black}}
\begin{multicols}{2}

\begin{tabular}{|c|c|c|c|}
\multicolumn{4}{c}{\textbf{Grade}} \\\hline
Exercise & Pages & Your Score & Max \\\hline
\multirow{2}{*}{1} & \multirow{2}{*}{2} & \multirow{2}{*}{\phantom{100100}} & \multirow{2}{*}{8} \\
& & & \\\hline
\multirow{2}{*}{2} & \multirow{2}{*}{3} & \multirow{2}{*}{\phantom{100100}} & \multirow{2}{*}{8} \\
& & & \\\hline
\multirow{2}{*}{3} & \multirow{2}{*}{4} & \multirow{2}{*}{\phantom{100100}} & \multirow{2}{*}{8} \\
& & & \\\hline
\multirow{2}{*}{4} & \multirow{2}{*}{5} & \multirow{2}{*}{\phantom{100100}} & \multirow{2}{*}{8} \\
& & & \\\hline
\multirow{2}{*}{5} & \multirow{2}{*}{6-7} & \multirow{2}{*}{\phantom{100100}} & \multirow{2}{*}{8} \\
& & & \\\hline
\multirow{2}{*}{\textbf{Total}} & \multirow{2}{*}{2-\pageref*{LastPage}} & \multirow{2}{*}{\phantom{100100}} & \multirow{2}{*}{40} \\
& & & \\\hline
\end{tabular}


\noindent\textbf{Additional Help and Resources.}
Did you use help and/or resources other than the textbook? Please indicate below.


\vspace{\fill}
\noindent\phantom{Dimitris}

\end{multicols}


%%%%%%%%%%%%%%%%%%%%%%%%%%%%%%%%%%%%%%%%%%%%%%%%%%%%%%%%%%%%%%%%%%%%%%%%%%%%%%%%%%%%%%%%%%%%%%%%%%%
%%%%%%%%%%%%%%%%%%%%%%%%%%%%%%%%%%%%%%%%%%%%%%%%%%%%%%%%%%%%%%%%%%%%%%%%%%%%%%%%%%%%%%%%%%%%%%%%%%%
%%%%%%%%%%%%%%%%%%%%%%%%%%%%%%%%%%%%%%%%%%%%%%%%%%%%%%%%%%%%%%%%%%%%%%%%%%%%%%%%%%%%%%%%%%%%%%%%%%%
%%%%%%%%%%%%%%%%%%%%%%%%%%%%%%%%%%%%%%%%%%%%%%%%%%%%%%%%%%%%%%%%%%%%%%%%%%%%%%%%%%%%%%%%%%%%%%%%%%%



\newpage
\section{Context-Free Grammar to Pushdown Automaton [8 points]}
Consider the context-free grammar \( G = (V, \Sigma, R, S) \) with
\( V = \{S, T\}, \Sigma = \{a,b,c\}\), 
and productions
\begin{displaymath}
\begin{array}{rcl}
S & \rightarrow & aSc \ \mid\ T \\
T & \rightarrow & bTc \ \mid\ \varepsilon
\end{array}
\end{displaymath}

\begin{enumerate}[label=(\roman*)]
   \item \textbf{[2 points]} Describe in words (and set notation) the language \(L(G)\).
    \item \textbf{[6 points]} Construct a \textbf{pushdown automaton (PDA)} \(M\) that accepts \(L(G)\). 
    Please use the construction that we have seen in class (also discussed in the book).
    As usual, the PDA should start with an empty stack and accept strings in the language with an empty stack.
\end{enumerate}






\newpage
\section{Pumping Lemma for Context-Free Languages [8 points]}\label{pump}
Let $\Sigma = \set{0, 1, 2, 3}$. 
Is the language $L_{1} = \{w\in\Sigma^* \mid \mbox{in $w$}$, 
the number of $0$'s equals the number of $1$'s, and the number of $2$'s equals 
the number of $3$'s$\}$ context-free? Show your answer is correct.







\newpage
\section{Assigned Reading [8 points]}
Read the Introduction, Sections 1 and 2, and Section 9 up to the end of 9.I of Alan Turing's 1936 paper 
``On Computable Numbers, with an Application to the Entscheidungsproblem'' that is available at: 
\href{https://www.diochnos.com/about/TuringCompute.pdf}{https://www.diochnos.com/about/TuringCompute.pdf}.
\begin{enumerate}[label=(\roman*)]
\item \textbf{[4 points]} Give an example of a computation that Turing would have probably believed a human would be unable to do without paper. Justify your answer.
\item \textbf{[4 points]} Is Turing more concerned with creating a machine that can simulate human computation or with creating a machine that humans can simulate? Justify your answer.
\end{enumerate}
\emph{Please answer the two questions above briefly, referring to and/or quoting the text for support. A well thought-out paragraph should be sufficient for each question.}








\newpage
\section{Assigned Reading [8 points]}
Read the article ``Who Can Name the Bigger Number?'' by Scott Aaronson, which you can find at:
\href{https://www.diochnos.com/teaching/CS3823/2025F/Aaronson99.pdf}{https://www.diochnos.com/teaching/CS3823/2025F/Aaronson99.pdf}. 
This article may include concepts we have not yet covered, but the introduction to them should be sufficiently self-contained.
\begin{enumerate}[label=(\roman*)]
\item \textbf{[4 points]} Going back to 5,000 B.C., name some advances in mathematics over time that have effectively allowed people to express bigger numbers than before the advent of those advances. 
You may do outside research to answer this question.\footnote{A few examples will suffice; do not go overboard and write a research paper.} 
Include at least one advance from Aaronson’s article.

\item \textbf{[4 points]} Would Aaronson say Turing machines are important even if they didn't have the immense real-world impact that they currently have? Why or why not?
\end{enumerate}








\newpage
\section{Designing a Turing Machine [8 points]}\label{design_tm}
Let $L_3$ be the language $\set{0^n1^n \mid n \ge 1}$ over $\Sigma = \set{0,1}$.
\begin{enumerate}[label=(\roman*)]
\item \textbf{[6 points]} Draw the state diagram for $M_1$, a deterministic Turing Machine that recognizes $L_1$. 
\emph{It will be assumed that all the transitions not depicted go to the reject state, which may or may not be drawn.}

\item \textbf{[2 points]} Is $M_1$, as constructed in the previous part, a decider? (If it is hard to answer this question, you should perhaps change your answer to the previous question; that is, redesign the machine.)
\end{enumerate}



\newpage
\begin{center}
\textbf{(extra space in case you need it for exercise \ref{design_tm})}
\end{center}





\end{document}
